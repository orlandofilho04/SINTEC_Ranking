\documentclass{modelo_resumo_simples}

\primeiroAutor{Orlando Soares de Santana Filho}
\abreviacaoPrimeiroAutor{SANTANA, O. S. F.}
\notaPrimeiroAutor{Estudante do curso de bacharelado em Sistemas de Informação, IF Goiano – Campus Ceres.}

\segundoAutor{Carlos Henrique Mota Martins}
\abreviacaoSegundoAutor{MARTINS, C. H. M.}
\notaReferenciaSegundoAutor{1}

\terceiroAutor{Matheus Oliveira Brandão}
\abreviacaoTerceiroAutor{BRANDÃO, M. O.}
\notaReferenciaTerceiroAutor{1}

\quartoAutor{Thiago Henrique Felix Cândido Ribeiro Conceição}
\abreviacaoQuartoAutor{CONCEIÇÃO, T. H. F. C. R.}
\notaReferenciaQuartoAutor{1}

\quintoAutor{Ronneesley Moura Teles}
\abreviacaoQuintoAutor{TELES, R. M.}
\notaQuintoAutor{Professor orientador, IF Goiano – Campus Ceres.}

\sextoAutor{Lucas José de Faria}
\abreviacaoSextoAutor{FARIA, L. J.}
\notaSextoAutor{Professor coorientador, IF Goiano – Campus Ceres.}
		
\titulo{Análise dos Estudantes no Ranking do SLab}

\begin{document}
	
	\construirtitulo

	\construirautores
	
	\begin{resumo}
	O emprego de questionários no formato de Quiz para avaliar e otimizar o processo de aquisição de conhecimento representa uma das estratégias implementadas pela Secretaria da Educação (Seduc) com o intuito de incentivar a motivação dos alunos. Contudo, torna-se imprescindível quantificar o progresso dos indivíduos que realizam tais questionários. Com base nesse contexto, surgiu a ideia de conduzir uma pesquisa envolvendo 8 estudantes que estejam devidamente matriculados no Instituto Federal Goiano Campus Ceres, submetendo-os a um Quiz diário durante 5 dias consecutivos. Para esse propósito, será utilizado o sistema de Quiz denominado "SLab", que foi desenvolvido e está em constante evolução pelo corpo docente do curso de Sistemas de Informação da mesma instituição de ensino em que os avaliados serão selecionados. A versão atual deste sistema, desenvolvida com as tecnologias HyperText Markup Language (HTML5), Cascading Style Sheets (CSS), Hypertext Preprocessor (PHP) e MySQL, apresenta funcionalidades que permitem a resposta a perguntas e realização de cálculos estatísticos, tais como média, moda, amplitude, desvio-padrão, entre outros. Além disso, a versão mais recente do sistema incorpora uma funcionalidade adicional, criada em sala de aula com o intuito de mensurar a evolução dos respondentes, que é o centro desse trabalho, validar sua utilização, tal funcionalidade é um ranking dos melhores desempenhos na plataforma, específico para o Quiz. Centrado na utilização do conceito de "Ranking" como métrica de avaliação do progresso dos participantes, foi proposto uma investigação destinada a analisar a classificação dos indivíduos em cada dia. Após a conclusão do período de 5 dias, foi possível realizar uma análise que indicou melhorias. Este aprimoramento refletiu-se no desempenho médio, evidenciando que todos os participantes experimentaram um aumento ou igualdade na média de respostas corretas em relação à primeira tentativa, onde na média da pontuação final, todos os participantes registraram um desempenho de pelo menos 60 de acertos nas questões, no entanto, na tentativa inicial, houve ocorrências em que apenas 20 das questões foram respondidas corretamente. Além disso, um dos estudantes, que inicialmente figurava entre as últimas posições nas duas primeiras tentativas do quiz, alcançou uma posição de destaque no pódio ao final do período. Esta ascensão sugere que sua classificação no ranking teve um impacto positivo em seu desempenho, caracterizando uma dinâmica de competição positiva. Assim foi possível identificar melhorias no domínio da estatística preditiva nos questionários por parte dos estudantes, bem como verificar que esses estudantes se mantiveram motivados devido à constante visualização de suas posições no ranking diário, sempre buscando aprimorar seu desempenho. Essa abordagem possibilitou fomentar uma competição saudável e otimizar o processo de aprendizagem.
	\end{resumo}
	
	\begin{palavras_chave}
	Quiz; Algoritmo; Programação; Sistemas; Ranking
	\end{palavras_chave}

\end{document}